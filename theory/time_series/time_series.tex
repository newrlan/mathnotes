\documentclass[../handbook.tex]{subfiles}
\graphicspath{{\subfix{../pics/}}}
\begin{document}

\chapter{Временные ряды}

В этой главе БЛА-БЛА-БЛА

\section{ARIMA}

\subsection{Модель авторегрессии (AR)}
\marginpar{AR --- autoregressive}
\emph{Авторегрессонная} модель временного ряда~$y_t$, это когда мы считаем, что~$y_t$ выражается через предыдущие наблюдения в линейном виде.

\marginpar{
    $\varepsilon_t$ - шум \\
    Всегда какой-то шум должен присутствовать. Хотя бы потому, что мы умеем точно все замерять.
}
\begin{equation}
    \label{eq:AR}
    y_t - \sum_{i=1} a_i y_{t-i} = \varepsilon_t
\end{equation}

\subsection{Характеристическое уравнение (I)}
\marginpar{I --- integrative}
Введем оператор $L(y_i) = y_{i-1}$ который получает предыдущий элемент. Тогда уравнение~\eqref{eq:AR} можно представить в виде

\begin{equation}
    \varepsilon_t = \left(y_t - \sum_{i=1}L^i y_t\right) =  (1 - \sum_{i=1}L^i) y_t
\end{equation}

\marginpar{
    Корни полинома~\eqref{eq:ts:character} говорят о стационарности. Если есть
    единичные корни, то процесс нестационарный, то есть зависит от времени.
    Если все корни больше нуля, то процесс стационарный, а если корни меньше
    нуля, процесс - взрывной.
}
Характеристическим уравнением временного ряда называется многочлен
\begin{equation}
    \label{eq:ts:character}
    a(z) = 1 - \sum_{i=1}z^i
\end{equation}

\marginpar{
    $(1 - L) \, y_t = y_t - y_{t-1} = \Delta \, y_t$ \\
    $(1 - L)^2 \, y_t = (1 - L) \Delta \, y_t = y_t - 2 y_{t-1} + y_{t-3} $ \\
    \dots \\
    $\Delta^k y_t = (1-L)^k y_t$ -- численная производная порядка~$k$.

}
\marginpar{Говорят, что ряд $y_t$ интегрированный ряд порядка~$k$ над стационарным рядом.}
В случае если есть k единичных корней, то полином можно представить в виде $a(z) = b(z) (1-z)^k$. Такое же верно и для оператора:
\begin{equation}
    a(L)\, y_t = b(L) (1 - L)^k \, y_t
\end{equation}
Теперь поскольку у b(z) нет единичных корней, он стационарен над рядом $(1-L)^k \, y_t$.

\subsection{Модель скользящего среднего MA}
\marginpar{MA --- moving average}
Модель скользящего среднего оперирует с ошибками предыдущих оценок. Общая формула такая
\begin{equation}
    \label{eq:ts:MA}
    y_t - (\mu + \sum_i \omega_i\varepsilon_{t-i}) = \varepsilon_t
\end{equation}
Здесь $\varepsilon_{t-i}$ это ошибки оценок предыдущих элементов временного ряда. Очевидно, что MA-модель является частным случаем AR модели, поскольку все~$\varepsilon_{t-i}$ выражаются через элементы ряда.

\subsection{Модель ARIMA(p, d, q)}
Это соединение трех предыдущих техник дает модель ARIMA --- авторегрессия, интегрирование, скользящее среднее.

\begin{equation}
    \label{eq:ts:ARIMA}
    y_t = \varepsilon_t + \sum_{i=1}^p a_i \Delta^d y_{t-i} + \sum_{i=1}^q\omega_i\varepsilon_{t-i}
\end{equation}

\marginpar{
    Модель ARIMA(p+d, 0, q) включает в себя ARIMA(p, d, q), поскольку уже в этом случае авторегрессия содержит все элементы чтобы вычислить производные порядка $d$.
}
Параметры в определении модели описывают:
\begin{itemize}
    \item $p$ -- количество компонентов авторегрессии
    \item $d$ -- порядок дифференцирования
    \item $q$ -- количество ошибок используемых в MA
\end{itemize}




\end{document}
