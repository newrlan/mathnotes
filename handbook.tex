% This is a modified version of the tufte-latex book example in which the title page and the contents page resemble Tufte's VDQI book, using Kevin Godby's code from this thread at https://groups.google.com/forum/#!topic/tufte-latex/ujdzrktC1BQ.
%
%% Unfortunately for the contents to contain
%% the "Parts" lines successfully, hyperref
%% needs to be disabled.
\documentclass{tufte-book}
% \usepackage[T2A]{fontenc} % Use 8-bit encoding that has 256 glyphs
\usepackage[OT1]{fontenc} % Use 8-bit encoding that has 256 glyphs
\usepackage[utf8]{inputenc} % Required for including letters with accents
\usepackage[russian]{babel}
\usepackage{graphicx}

\title{Вероятность и \\ статистика \\ \Huge{Поваренная книга}}

\author{Nurlan Sadykov}

% Абстракт
% Заметки по решению кубика рубика с помощью теории групп.

\begin{document}

\frontmatter


\maketitle  

\chapter{Probability}

\chapter{Statistics}

\section{Cookbook}

\begin{enumerate}
    \item \emph{Репозиторий.} Анализ самостоятельного датасета лучше поместить
        в отдельную папку.

    \item \emph{Описальные статистики} стоит просмотреть чтобы понять качество
        данных и исключить сильные аномалии.

    \item Построить \emph{гистограмму} чтобы понять распределение и проверить
        на выбросы.\sidenote{Иногда удобно просматривать логарифмированную
        фичу. Если фича сильно отстоит от нуля, ее лучше переместить в ноль.} 

\end{enumerate}

\end{document}
